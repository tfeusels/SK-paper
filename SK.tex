%\documentclass[preprint,5p]{elsarticle}
\documentclass[preprint,3p]{elsarticle}

%\usepackage{ecrc}
%% The ecrc package defines commands needed for running heads and logos.
%% For running heads, you can set the journal name, the volume, the starting page and the authors

%\volume{00}
%\firstpage{1}
%% Give the name of the journal
\journal{Nuclear Instruments and Methods}



%% The choice of journal logo is determined by the \jid and \jnltitlelogo commands.
%% A user-supplied logo with the name <\jid>logo.pdf will be inserted if present.
%% e.g. if \jid{yspmi} the system will look for a file yspmilogo.pdf
%% Otherwise the content of \jnltitlelogo will be set between horizontal lines as a default logo

%% Give the abbreviation of the Journal.
%\jid{procs}

%% Give a short journal name for the dummy logo (if needed)
%\jnltitlelogo{Procedia Computer Science}

%% Hereafter the template follows `elsarticle'.
%% For more details see the existing template files elsarticle-template-harv.tex and elsarticle-template-num.tex.

%% Elsevier CRC generally uses a numbered reference style
%% For this, the conventions of elsarticle-template-num.tex should be followed (included below)
%% If using BibTeX, use the style file elsarticle-num.bst

%% End of ecrc-specific commands
%%%%%%%%%%%%%%%%%%%%%%%%%%%%%%%%%%%%%%%%%%%%%%%%%%%%%%%%%%%%%%%%%%%%%%%%%%
\usepackage{geometry}
\usepackage[parfill]{parskip}   
\usepackage{graphicx}
\usepackage{amsmath}
\usepackage[]{color}
\usepackage[separate-uncertainty=true, range-phrase = --,
range-units=single, per-mode=symbol]{siunitx}
\usepackage{booktabs}

\setlength{\textwidth}{7.0in}
\setlength{\textheight}{9in} 
\setlength{\oddsidemargin}{-.25in}
\setlength{\evensidemargin}{-0.25in} 
\setlength{\topmargin}{-0.6in}
\setlength{\parskip}{3ex plus 0.5ex minus 0.2ex}

%% The amssymb package provides various useful mathematical symbols
\usepackage{amssymb}
%% The amsthm package provides extended theorem environments
\usepackage{amsthm}

%% The lineno packages adds line numbers. Start line numbering with
%% \begin{linenumbers}, end it with \end{linenumbers}. Or switch it on
%% for the whole article with \linenumbers after \end{frontmatter}.
\usepackage{lineno}

%% natbib.sty is loaded by default. However, natbib options can be
%% provided with \biboptions{...} command. Following options are
%% valid:

%%   round  -  round parentheses are used (default)
%%   square -  square brackets are used   [option]
%%   curly  -  curly braces are used      {option}
%%   angle  -  angle brackets are used    <option>
%%   semicolon  -  multiple citations separated by semi-colon
%%   colon  - same as semicolon, an earlier confusion
%%   comma  -  separated by comma
%%   numbers-  selects numerical citations
%%   super  -  numerical citations as superscripts
%%   sort   -  sorts multiple citations according to order in ref. list
%%   sort&compress   -  like sort, but also compresses numerical citations
%%   compress - compresses without sorting
%%
\biboptions{comma,round,sort&compress,numbers}


% if you have landscape tables
\usepackage[figuresright]{rotating}

% put your own definitions here:
%   \newcommand{\cZ}{\cal{Z}}
%   \newtheorem{def}{Definition}[section]
%   ...

% add words to TeX's hyphenation exception list
\hyphenation{author another created financial paper re-commend-ed Post-Script}

% declarations for front matter

\begin{document}

\graphicspath{{figs/}}

\begin{frontmatter}

%% Title, authors and addresses

%% use the tnoteref command within \title for footnotes;
%% use the tnotetext command for the associated footnote;
%% use the fnref command within \author or \address for footnotes;
%% use the fntext command for the associated footnote;
%% use the corref command within \author for corresponding author footnotes;
%% use the cortext command for the associated footnote;
%% use the ead command for the email address,
%% and the form \ead[url] for the home page:
%%
\title{Super-Kamiokande Photo Multiplier Tube Calibration}


\author[ins:ubc,ins:triu]{S. Berkman}
\author[ins:ubc,ins:triu]{T. Feusels}
\ead{tfeusels@triumf.ca}
%\ead[url]{home page}
%\fntext[label4]{University of British Columbia}
\cortext[cor1]{}
\author[ins:triu]{W. Faszer}
\author[ins:triu]{D. Morris}
\author[ins:triu,ins:kav]{M. Hartz}
\author[ins:triu]{A. Konaka}
\author[ins:triu]{T. Lindner}
\author[]{B. Krupicz}
\author[]{A. Jaffray}
\author[]{H. Kugel}
\author[]{P. de Perlo}
\author[]{J. Linquiao Liu}
\author[]{P. Lu}
\author[ins:triu]{A. Miller}
\author[]{C. Nantais}
\author[]{C. Reithmeier}
\author[ins:triu]{F. Reti\`ere}
\author[]{S. Sourkar}
\author[ins:triu]{M. Scott}
\author[]{S. Stephen-Whale}
\author[]{N. Smit-Anseeuw}
\author[]{H. Tanaka}
\author[ins:ubc,ins:triu]{S. Tobayama}
\author[ins:triu]{P. Vincent}
\author[]{M. Walters}
\author[]{M. Wilking}
\author[ins:triu]{S. Yen}
\author[]{A. Zimmer}
\author[]{M. Walters}
\author[]{R. Sugimoto}
\author[]{K. Gin Sie Xie}
\author[]{J. Kour RA}
\author[ins:kash]{Y. Nishimura}
\author[ins:toky]{Y. Suda}
\author[ins:toky2]{Y. Okajima}
\author[ins:okaj,ins:kav]{Y. Koshio}
\author[ins:kamio,ins:oxf]{A. Takenaka}
\author[ins:kamio,ins:oxf]{S. Moriyama}


\address[ins:ubc]{University of British Columbia, Department of Physics and Astronomy, Vancouver, British Columbia, Canada}
\address[ins:triu]{TRIUMF, Vancouver, British Columbia, Canada}
\address[ins:kav]{Kavli Institute for the Physics and Mathematics of the Universe (WPI), The University
of Tokyo Institutes for Advanced Study, University of Tokyo, Kashiwa, Chiba, Japan}
\address[ins:kash]{Research Center for Cosmic Neutrinos, Institute for Cosmic Ray
Research, University of Tokyo, Kashiwa, Chiba 277-8582, Japan}
\address[ins:toky]{Department of Physics, University of Tokyo, Bunkyo, Tokyo 113-0033, Japan}
\address[ins:toky2]{Department of Physics, Tokyo Institute of Technology, Meguro, Tokyo 152-8551, Japan}
\address[ins:okaj]{Department of Physics, Okayama University, Okayama, Okayama 700-8530, Japan}
\address[ins:kamio]{Kamioka Observatory, Institute for Cosmic Ray Research, University of Tokyo, Kamioka, Gifu 506-1205, Japan}
\address[ins:oxf]{STFC, Rutherford Appleton Laboratory, Harwell Oxford, and Daresbury Laboratory, Warrington, OX11 0QX, United Kingdom}
%\dochead{Short communication}
%% Use \dochead if there is an article header, e.g. \dochead{Short communication}

%% use optional labels to link authors explicitly to addresses:
%% \author[label1,label2]{<author name>}
%% \address[label1]{<address>}
%% \address[label2]{<address>}

%UBC, TRIUMF and visitors: \\
%Sophie Berkman, Ben Krupicz, Patrick de Perio, Wayne Faszer, Tom Feusels, Mark Hartz, 
%Alexander Jaffray, Akira Konaka, Harish Kugel,
%Thomas Lindner, James Linqiao Liu, Philip Lu, Andy Miller, David Morris, Corina Nantais, 
%Yasuhiro Nishimura,
%Carl Reithmeier, Fabrice Retière, Sourav Sarkar, Mark Scott, Shaun Stephens-Whale, 
%Nils Smit-Anseeuw, Yusuke Suda, Hiro Tanaka, Shimpei Tobayama,
%Peter Vincent, Michael Walters, Michael Wilking, Stan Yen, Aaron Zimmer\\
%Y.Nishimura and Okajima
%Yusuke Koshio (Okayama Uni)
%Rika Sugimoto
%TAKENAKA akira
%Shigetaka MORIYAMA <moriyama@icrr.u-tokyo.ac.jp>
%Kevin Gin Sing Xie(Waterloo)
%Jashan Kaur RA (uni Winnipeg)

\author{}

\address{}

\begin{abstract}
%% Text of abstract
\end{abstract}

\begin{keyword}
%% keywords here, in the form: keyword \sep keyword

%% MSC codes here, in the form: \MSC code \sep code
%% or \MSC[2008] code \sep code (2000 is the default)

\end{keyword}

\end{frontmatter}

%%
%% Start line numbering here if you want
%%
\linenumbers

%% main text
\section{Introduction: Motivation}

Angular response in the water and impact from simulation


\section{Results}

\subsection{Vertical injection in air}
Without acrylic cover

2nd PMT

Gain, relative efficiency, transit time (spread)

Magnetic Field


\subsection{Other Studies}

normal incident scan

wavelength and polarization dependence

measurement in the water


\section{Discussion}

Explanation of the observed position dependence in gain/efficiency/time

Simulation studies?



\input{./Appendix_.tex}


%% References
%%
%% Following citation commands can be used in the body text:
%% Usage of \cite is as follows:
%%   \cite{key}         ==>>  [#]
%%   \cite[chap. 2]{key} ==>> [#, chap. 2]
%%

%% References with BibTeX database:

%\bibliographystyle{elsarticle-num}
%\bibliography{<your-bib-database>}

%% Authors are advised to use a BibTeX database file for their reference list.
%% The provided style file elsarticle-num.bst formats references in the required Procedia style

%% For references without a BibTeX database:

% \begin{thebibliography}{00}

%% \bibitem must have the following form:
%%   \bibitem{key}...
%%

% \bibitem{}

% \end{thebibliography}

\end{document}
